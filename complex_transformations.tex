\documentclass{article}
\usepackage{amsmath}
\usepackage{amssymb}

\title{Complex Transformation IFS for Tiling Levy Curve}
\author{}
\date{}

\begin{document}
\maketitle

The Iterated Function System (IFS) for the Levy curve is defined by the following set of complex transformations:

\begin{equation}
\begin{cases} 
f_1(z) = (0.5 - 0.5i)(z - (-0.5 + 0.5i)) + (-0.5 + 0.5i) \\
f_2(z) = (0.5 - 0.5i)e^{i\pi/2}(z - (-0.5 + 0.5i)) + (0.5 - 0.5i) + (-0.5 + 0.5i)
\end{cases}
\end{equation}

The completed curve is rotated by multiples of $90^\circ$ to construct the unit tile.
Tiles are then shifted by increments of one unit to tile the plane.

The rotations and shifts are applied according to:

\begin{equation}
\begin{cases}
R_{k,a,b}(z) = ze^{i\pi k/2} + (a + bi) \\
\text{where } k \in \{0,1,2,3\}, \text{ and } a,b \in \mathbb{Z}
\end{cases}
\end{equation}

The complete Levy Mosaic is generated by applying $f_1$ and $f_2$ to the initial point, followed by all possible combinations of rotations and shifts $R_{k,a,b}$.

\section*{Explanation of Tiling}
The tiling system works in two distinct steps:

\begin{enumerate}
    \item \textbf{Initial Transformations:} First, we apply $f_1$ and $f_2$ iteratively to create the Levy curve.
    
    \item \textbf{Symmetry Rotations:} After generating all points, we rotate the curve by multiples of $90^\circ$:
    \begin{itemize}
        \item $k=0$ gives $e^{0} = 1$ (no rotation)
        \item $k=1$ gives $e^{i\pi/2} = i$ (90° rotation)
        \item $k=2$ gives $e^{i\pi} = -1$ (180° rotation)
        \item $k=3$ gives $e^{3i\pi/2} = -i$ (270° rotation)
    \end{itemize}
    
    These four rotations create the square symmetry of the basic tile.
\end{enumerate}

Finally, we create the tiling pattern by shifting each rotated point by integer combinations $(a,b)$ where $a,b \in \mathbb{Z}$, allowing the pattern to extend infinitely in all directions.

\end{document}